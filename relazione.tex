\documentclass[12pt]{article}
\usepackage{graphicx}
\usepackage[utf8]{inputenc}
\usepackage{geometry}
\usepackage{fancyvrb}
\usepackage{enumitem}  % Include this line in the preamble

\geometry{
    top=1cm,
    bottom=1.5cm,
    left=1.5cm,
    right=1.5cm
}

\title{robotSpace: Una libreria Java con applicazione terminale per la simulazione di uno sciame di robot}
\author{ Redacted }
\date{27 Giugno 2023}

\begin{document}

    \maketitle

    \section{Introduzione}
    Il progetto robotSpace si prefigge di realizzare una libreria Java con applicazione terminale per la simulazione di uno sciame di robot. I robot operano in un ambiente bidimensionale di dimensioni illimitate, dove possono muoversi liberamente.

    \section{Dettagli Tecnici}
    Il progetto è sviluppato in Java 19 con l'utilizzo del sistema di gestione dei progetti Gradle, assicurando una struttura del codice pulita e facilmente estendibile, i vari test sono inclusi nel progetto. Il progetto richiede pertanto un'installazione funzionante della JDK di Java 19 nel sistema in cui viene eseguito.

    \section{Responsabilità assegnate}
    \begin{itemize}
        \item La classe App interagisce con RobotSpaceService, e, dopo aver analizzato ed elaborato i parametri in riga di comando passati a ``gradle run`` si occupa di inizializzare correttamente lo stato di quest'ultimo.
        \item RobotSpaceService gestisce la simulazione dell'ambiente. È responsabile per l'inizializzazione e la gestione di parser, programma, ambiente e controller dei Robot.
        \item RobotController rappresenta un contenitore che contiene tutti i robot e permette alla classe RobotSpaceService di interagire con tutti i Robot allo stesso momento. È responsabile dell'orchestrazione di tutti i Robot, della loro programmazione, dell'input dei dati ambientali presso i Robot.
        \item RobotProgram rappresenta un "compilatore" rudimentale che trasforma il programma da testo a istanze di varie classi rappresentanti tutte le varie istruzioni (rappresentazione intermedia) organizzate in maniera da renderne l'esecuzione molto più semplice. È stato scritto per essere facilmente espandibile. RobotSpaceService ne fa uso per "compilare" il programma. Utilizza una classe di aiuto ParsingBodyStack per implementare il parsing corretto di loop annidati.
        \item Entity è la classe base di Robot, e si occupa di gestire posizione, rotazione, velocità e altri fattori di movimento, astrae tutta la fisica e matematica necessaria per il movimento, in maniera che l'implementazione del Robot risulti più semplice.
        \item Robot rappresenta un singolo robot all'interno della simulazione, ed è basato sulla classe Entity. È responsabile della manutenzione del suo stato attuale e dello stato del suo processore, che viene astratto tramite l'utilizzo di un'interfaccia RobotContext i cui metodi servono a controllare l'entità Robot. Utilizza una classe di aiuto LoopBodyStack per implementare l'esecuzione di corpi annidati di loop e per tenere traccia dei vari instruction pointer all'interno dei suddetti corpi.
        \item Rectangle e Circle sono classi che implementano l'interfaccia Shape, e sono utilizzate per astrarre i diversi comportamenti a livello matematico che il codice ha per effettuare il calcolo delle aree. Implementano diverse funzioni per ottenere posizioni casuali all'interno delle figure e per controllare dove si trovano punti arbitrari in relazione alla figura stessa. Queste classi sono pertanto responsabili per l'orientamento ambientale dei Robot.
    \end{itemize}

    \section{Istruzioni}
    Per effettuare una compilazione del programma è sufficiente eseguire ``gradle build`` nella cartella principale in cui si trova (la JDK di Java 19 deve essere installata). Per eseguire il programma basta eseguire ``gradle run``. I vari parametri che il programma si aspetta saranno stampati a schermo se non vengono inseriti correttamente. Lo stato dei vari Robot all'interno dell'ambiente verrá stampato nel terminale ad ogni tick di processore.

    \begin{description}[leftmargin=2em]
        \item[Comando di esempio:]
    \end{description}
    \begin{Verbatim}
    gradle run --args="
    2
    string
    'Z2 CIRCLE 5 6 5\nZ1 RECTANGLE 1 1 2 2'
    string
    'SIGNAL Z2\nSIGNAL Z1\nMOVE 1 0 1\nUNSIGNAL Z1\nCONTINUE 5\nMOVE -1 -1 2\nCONTINUE 7'
    1000
    1000"
    \end{Verbatim}

    \begin{description}[leftmargin=2em]
        \item[Istruzioni di esecuzione e formattazione degli argomenti del programma:]
    \end{description}
    \begin{Verbatim}
    gradle run --args="<numRobots> <envType> <env> <progType> <prog> <tpi> <stpi>"

    Arguments:
    numRobots: Number of robots to simulate. Robots will be placed randomly in the
    grid according to the environment defined. If no environment is defined, all
    the robots will start at a random position, with x and y both being between
    (-10, 10).

    envType: Type of environment, string or file.

    env: If the type was file, an absolute path to the file containing it.
    Otherwise a string, use \n to separate new lines.

    progType: Type of program, string or file.

    prog: If the type was file, an absolute path to the file containing the program.
    Otherwise a string, use \n to separate new lines.

    tpi: Time per instruction (ms), how much time the program should sleep before
    executing the next instruction, can be 0, it will influence how the CONTINUE
    instruction works. Should be 1000ms by default.

    stpi: Simulation time per instruction (ms), how much time should pass in the
    simulation for every instruction executed.
    \end{Verbatim}

    \begin{description}[leftmargin=2em]
        \item[Test JUnit:]
    \end{description}

    I test JUnit coprono oltre l'80\% del codice e il 100\% delle classi principali, rendendo semplice l'estensibilità e il mantenimento del progetto, assicurando un elevato grado di affidabilità del software. I test possono essere eseguiti utilizzando il comando seguente:

    \begin{Verbatim}
    gradle test
    \end{Verbatim}

    \section{Conclusione}
    Il progetto robotSpace è una sfida che pone l'accento su concetti chiave della programmazione orientata agli oggetti e la gestione dei progetti, creando una simulazione interessante e dinamica di uno sciame di Robot.

\end{document}